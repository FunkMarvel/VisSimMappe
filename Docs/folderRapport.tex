\documentclass[norsk, doc, 12pt, a4paper]{apa7}  % defines the basic parameters of the document

% if you want a single-column, remove reprint

% allows special characters (including æøå)
\usepackage[norsk]{babel}
\usepackage[utf8]{inputenc}

\usepackage{csquotes}
\usepackage[style=apa, backend=biber]{biblatex}

%% note that you may need to download some of these packages manually, it depends on your setup.
%% I recommend downloading TeXMaker, because it includes a large library of the most common packages.

\usepackage{physics,amssymb}  % mathematical symbols (physics imports amsmath)
\usepackage{graphicx}         % include graphics such as plots
\usepackage{xcolor}           % set colors
\usepackage{hyperref}         % automagic cross-referencing (this is GODLIKE)
\usepackage{tikz}             % draw figures manually
\usetikzlibrary{tikzmark}
\usepackage{listings}         % display code
\usepackage{subfigure}        % imports a lot of cool and useful figure commands
\usepackage{cprotect}
\usepackage{float}
\usepackage{caption}

%\setlength{\parindent}{0px}

%Ta med disse kommandoene
\newcommand{\nnode}[2]{\node (#1) at (#2) {#1};}    %Her defineres nnode kommandoen
\newcommand{\relnode}[2]{\draw (#2) node[fill,circle,scale=0.6]{} (#2);\node (#1) at (#2) {};}  %Her defineres relnode kommandoen

% defines the color of hyperref objects
% Blending two colors:  blue!80!black  =  80% blue and 20% black
\hypersetup{ % this is just my personal choice, feel free to change things
    colorlinks,
    linkcolor={red!50!black},
    citecolor={blue!50!black},
    urlcolor={blue!80!black}}
%
%% Defines the style of the programming listing
%% This is actually my personal template, go ahead and change stuff if you want
\lstnewenvironment{python}{
	\lstset{ %
		inputpath=,
		backgroundcolor=\color{white!95!black},
		basicstyle={\ttfamily\scriptsize},
		commentstyle=\color{orange},
		language=Python,
		%numbers=left,
		%stepnumber=1,
		morekeywords={True,False},
		tabsize=4,
		stringstyle=\color{green!55!black},
		frame=single,
		keywordstyle=\color{blue},
		showstringspaces=false,
		columns=fullflexible,
		keepspaces=true,
		upquote=true}
}{}

\lstnewenvironment{cpp}{
	\lstset{ %
		inputpath=,
		backgroundcolor=\color{white!95!black},
		basicstyle={\ttfamily\scriptsize},
		commentstyle=\color{orange},
		language=C++,
		%numbers=left,
		%stepnumber=1,
		morekeywords={True,False},
		tabsize=4,
		stringstyle=\color{green!55!black},
		frame=single,
		keywordstyle=\color{blue},
		showstringspaces=false,
		columns=fullflexible,
		keepspaces=true,
		upquote=true}
}{}

\lstnewenvironment{csharp}{
	\lstset{ %
		inputpath=,
		backgroundcolor=\color{white!95!black},
		basicstyle={\ttfamily\scriptsize},
		commentstyle=\color{orange},
		language=[Sharp]C,
		%numbers=left,
		%stepnumber=1,
		morekeywords={True,False},
		tabsize=4,
		stringstyle=\color{green!55!black},
		frame=single,
		keywordstyle=\color{blue},
		showstringspaces=false,
		columns=fullflexible,
		keepspaces=true,
		upquote=true}
}{}

\lstnewenvironment{sql}{
	\lstset{ %
		inputpath=,
		backgroundcolor=\color{white!95!black},
		basicstyle={\ttfamily\scriptsize},
		commentstyle=\color{orange},
		language=SQL,
		%numbers=left,
		%stepnumber=1,
		morekeywords={True,False},
		tabsize=4,
		stringstyle=\color{green!55!black},
		frame=single,
		keywordstyle=\color{blue},
		showstringspaces=false,
		columns=fullflexible,
		keepspaces=true,
		upquote=true}
}{}

\lstnewenvironment{mongodb}{
	\lstset{ %
		inputpath=,
		backgroundcolor=\color{white!95!black},
		basicstyle={\ttfamily\scriptsize},
		commentstyle=\color{orange},
		language=bash,
		%numbers=left,
		%stepnumber=1,
		morekeywords={True,False},
		tabsize=4,
		stringstyle=\color{green!55!black},
		frame=single,
		keywordstyle=\color{blue},
		showstringspaces=false,
		columns=fullflexible,
		keepspaces=true,
		upquote=true}
}{}

\lstnewenvironment{php}{
	\lstset{ %
		mathescape=false,
		inputpath=,
		backgroundcolor=\color{white!95!black},
		basicstyle={\ttfamily\scriptsize},
		commentstyle=\color{orange},
		language=php,
		%numbers=left,
		%stepnumber=1,
		morekeywords={True,False},
		tabsize=4,
		stringstyle=\color{green!55!black},
		frame=single,
		keywordstyle=\color{blue},
		showstringspaces=false,
		columns=fullflexible,
		keepspaces=true,
		upquote=true}
}{}


%\lstset{literate=
%  {á}{{\'a}}1 {é}{{\'e}}1 {í}{{\'i}}1 {ó}{{\'o}}1 {ú}{{\'u}}1
%  {Á}{{\'A}}1 {É}{{\'E}}1 {Í}{{\'I}}1 {Ó}{{\'O}}1 {Ú}{{\'U}}1
%  {à}{{\`a}}1 {è}{{\`e}}1 {ì}{{\`i}}1 {ò}{{\`o}}1 {ù}{{\`u}}1
%  {À}{{\`A}}1 {È}{{\'E}}1 {Ì}{{\`I}}1 {Ò}{{\`O}}1 {Ù}{{\`U}}1
%  {ä}{{\"a}}1 {ë}{{\"e}}1 {ï}{{\"i}}1 {ö}{{\"o}}1 {ü}{{\"u}}1
%  {Ä}{{\"A}}1 {Ë}{{\"E}}1 {Ï}{{\"I}}1 {Ö}{{\"O}}1 {Ü}{{\"U}}1
%  {â}{{\^a}}1 {ê}{{\^e}}1 {î}{{\^i}}1 {ô}{{\^o}}1 {û}{{\^u}}1
%  {Â}{{\^A}}1 {Ê}{{\^E}}1 {Î}{{\^I}}1 {Ô}{{\^O}}1 {Û}{{\^U}}1
%  {œ}{{\oe}}1 {Œ}{{\OE}}1 {æ}{{\ae}}1 {Æ}{{\AE}}1 {ß}{{\ss}}1
%  {ű}{{\H{u}}}1 {Ű}{{\H{U}}}1 {ő}{{\H{o}}}1 {Ő}{{\H{O}}}1
%  {ç}{{\c c}}1 {Ç}{{\c C}}1 {ø}{{\o}}1 {å}{{\r a}}1 {Å}{{\r A}}1
%  {€}{{\euro}}1 {£}{{\pounds}}1 {«}{{\guillemotleft}}1
%  {»}{{\guillemotright}}1 {ñ}{{\~n}}1 {Ñ}{{\~N}}1 {¿}{{?`}}1
%}

\newcommand{\set}[1]{\ensuremath{\left\{#1\right\}}}
\newcommand{\tuple}[1]{\ensuremath{\left\langle #1 \right\rangle}}
\newcommand{\imp}{\ensuremath{\rightarrow}}

\newcommand{\ceil}[1]{\ensuremath{\lceil #1 \rceil}}
\newcommand{\floor}[1]{\ensuremath{\lfloor #1 \rfloor}}

\usepackage{thmtools}
\DeclareMathOperator{\nullspace}{Nul}
\DeclareMathOperator{\collspace}{Col}
\DeclareMathOperator{\rref}{Rref}
%%\DeclareMathOperator{\dim}{Dim}

 % "meq": must be equal
\newcommand{\meq}{\overset{!}{=}}
\newcommand\numberthis{\addtocounter{equation}{1}\tag{\theequation}}

\newcommand{\R}{\mathbb{R}}
\newcommand{\N}{\mathbb{N}}
\newcommand{\Z}{\mathbb{Z}}
\newcommand{\Q}{\mathbb{Q}}
\newcommand{\C}{\mathbb{C}}
\newcommand*\Heq{\ensuremath{\overset{\kern2pt L'H}{=}}}
\usepackage{bm}
\newcommand{\uveci}{{\bm{\hat{\textit{\i}}}}}
\newcommand{\uvecj}{{\bm{\hat{\textit{\j}}}}}

\DeclareRobustCommand{\uvec}[1]{{%
  \ifcsname uvec#1\endcsname
     \csname uvec#1\endcsname
   \else
    \bm{\hat{\mathbf{#1}}}%
   \fi
}}
\usepackage[binary-units=true]{siunitx}

\newcommand{\twopartdef}[4]
{
	\left\{
		\begin{array}{ll}
			#1 & \mbox{if } #2 \\
			#3 & \mbox{if } #4
		\end{array}
	\right.
}

\makeatletter
\newcommand*{\balancecolsandclearpage}{%
  \close@column@grid
  \cleardoublepage
  \twocolumngrid
}
\makeatother

\AtBeginEnvironment{align}{\setcounter{equation}{0}}
\newcounter{subproject}
\renewcommand{\thesubproject}{\alph{subproject}}
\newenvironment{subproj}{
\begin{description}
	\item[\refstepcounter{subproject}(\thesubproject)]
}{\end{description}}

\addbibresource{referanser.bib}

\title{Mappeoppgave Visualisering og Simulering}   % self-explanatory
\author{Anders P. Åsbø | kandidat.nr.: 843}               % self-explanatory
\affiliation{Høgskolen i Innlandet}
\date{\today}                             % self-explanatory
\shorttitle{2VSIM101 | kandidat.nr.: 843}
%\noaffiliation                            % ignore this


\abstract{ }

\begin{document}
\maketitle                                % creates the title, author, date
\tableofcontents

\section{Introduksjon}
Et viktig aspekt av moderne beredskap er forståelsen av ekstremvær og dets effekt på lokal natur. Derfor er det nødvendig å kunne lage digitale representasjoner av virkelige terreng, samt simulere fysikk ved bruk av datamaskin på slike representasjoner. I denne rapporten presenterer jeg en metode for å modellere, samt simulere effekten av nedbør på terreng konstruert fra punktskydata.

Rapporten fokuserer på hvordan lage en regulært indeksert trekantflate som representerer terrenget, samt hvordan bruke B-Splines til å kartlegge vassdrag som dannes ved ekstrem nedbør, og hvordan simulere effekten et slik vassdrag har på løsmateriale.

\section{Metode}
\subsection{Punktsky og triangulering}
\subsubsection{Vertekser} \label{M:1:1}
Moderne målinger av terreng gjøres med LiDAR, og resulterer i rådata i form av en uorganisert samling punkter i \(\R^{3}\) relativt til et valgt koordinatsystem \parencite{bergerSurveySurfaceReconstruction2017}. Jeg har valgt å laste ned punktdata fra Kartverket sin database '\url{hoydedata.no}' i '.laz'-format, som jeg så konverterer til en rentekst-fil ved hjelp av programvaren 'LASzip' \parencite{isenburgLASzip2019}. Den resulterende tekstfilen inneholder da \(n\) antall punkter fordelt på formen
\begin{align*}
 x_{0} \quad &y_{0} \quad z_{0} \\
 x_{1} \quad &y_{1} \quad z_{1} \\
 ... \quad &... \quad ... \\
 x_{n-1} \quad &y_{n-1} \quad z_{n-1}
\end{align*}
hvor hver linje svarer til et punkt. For å kunne visualisere punktskyen som en sammenhengende overflate, så konstruerer jeg et regulært rutenett i \(xy\)-planet med dimensjoner \(\textbf{width} = x_{\text{max}} - x_{\text{min}}\) og \(\textbf{height} = y_{\text{max}} - y_{\text{min}}\). Hvor \(x_{\text{max}}\), \(x_{\text{min}}\), \(y_{\text{max}}\) og \(y_{\text{min}}\) er henholdsvis største og minste verdi for \(x\)- og \(y\)-koordinatene til punktene. Jeg velger så at hver rute i det regulære rutenettet er et kvadrat med areal \(\SI{10}{\metre\squared}\) (steglengde \(h = \SI{10}{\metre}\)), slik at \(n_{x} = \lceil \textbf{width}/h \rceil\) og \(n_{y} = \lceil \textbf{height}/h \rceil\) er henholdsvis antall ruter i \(x-\) og \(y-retning\). Hvert punkt kan så sorteres inn i ruten som dekker \(x\)- og \(y\)-koordinatene dens ved å regne ut rad- og kolonneindeks \(i\) og \(j\) i rutenettet som
\begin{align*}
	i &= \lfloor \frac{x - x_{\text{min}}}{h} \rfloor \\
	j &= \lfloor \frac{y - y_{\text{min}}}{h} \rfloor.
\end{align*}
For hver rute kan det regnes ut en gjennomsnittlig \(z\)-koordinat ved å ta gjennomsnittet av \(z\)-koordinatene til alle punktene med tilsvarende rad- og kolonneindeks
\begin{align*}
	\bar{z}_{i,j} &= \frac{1}{n_{i,j}}\sum_{k = 0}^{n_{i,j}}z_{i,j,k},
\end{align*}
hvor \(n_{i,j}\) er antall punkter innenfor ruten som tilsvarer rad \(i\) kolonne \(j\). Hvis en rute skulle vise seg å ikke inneholde punkter, så kan gjennomsnittshøyden i den ruten regnes som gjennomsnittet av gjennomsnittshøydene til naborutene
\begin{align*}
	\bar{z}_{i,j} &= \frac{1}{8}\sum_{k=i-1}^{i+1}\sum_{\substack{l=j-1 \\ k\neq i \lor l\neq j}}^{j+1} \bar{z}_{k,l}.
\end{align*}
Hvis dette må regnes ut for en rute på kanten av rutenettet, så må alle naboer som ikke eksisterer ekskluderes fra beregningen over.
Et sett med regulært fordelte punkter i \(xy\)-planet, sentrert rundt origo, kan skrives til fil på formatet
\begin{align*}
	&n_{x}\cdot n_{y} \\
	&(i\cdot h - n_{x}\cdot h/2,\quad \bar{z}_{i,j} - (z_{\text{max} - z_{\text{min}}})/2,\quad j\cdot h - n_{y}\cdot h/2) \\
	&...
\end{align*}
for alle \(j=0,1,2,...,n_{y}-1\) for alle \(i=0,1,2,...,n_{x}-1\). Første linjen i filen inneholder antall punkter. Videre er hver koordinat translatert slik at origo er sentrert, og \(y\)- og \(z\)- koordinaten er byttet om for innlesning i Unity \parencite{UnityEngine2023} som bruker et venstrehendt koordinatsystem med positiv \(y\)-akse som oppoverretning.

\subsubsection{Indeksering}
For å kunne tegne en trekantflate av de regulært fordelte punktene i underseksjonen '\hyperref[M:1:1]{Vertekser}', så må det konstrueres en regulær triangulering med punktene som vertekser i trekanter. Dette gjøres ved å ta for seg alle kvadrater hjørner \(\vec{v}_{j + i\cdot n_{y}}\), \(\vec{v}_{(j+1) + i\cdot n_{y}}\), \(\vec{v}_{j + (i+1)\cdot n_{y}}\) og \(\vec{v}_{(j+1) + (i+1)\cdot n_{y}}\), hvor \(\vec{v}_{k}\) er de regulært fordelte punktene, og \(j=0,1,2,...,n_{y}-1\) for alle \(i=0,1,2,...,n_{x}-1\). Videre er det to trekanter \(T_{2(j + i\cdot (n_{y}-1))}\) og \(T_{2(j + i\cdot (n_{y}-1)) + 1}\) per kvadrat.



\subsection{Ball på trekantflate}
\subsection{B-Splines og simulering av vassdrag}
\parencite[s.7]{alexanderMovingBouldersFlash2016}
\section{Resultater}
\section{Diskusjon}
\section{Konklusjon}

\printbibliography

\end{document}